
\documentclass[10pt]{article}

\usepackage{listings}
\usepackage{geometry}
\usepackage{parskip}
\usepackage{xcolor}

\geometry{a4paper, left=1.5cm, right=1.5cm}

\author{David Burian}
\date{\today}
\title{Vector Space Model}

\definecolor{lightlightgray}{HTML}{efefef}
\lstset{backgroundcolor=\color{lightlightgray},framerule=0pt}

\begin{document}
\maketitle


\section{Instalace}

Moje řešení je implementováno v pythnu s verzí \verb|3.10.2|. Všechny potřebné
balíčky jsou k nahlédnutí v \verb|requirements.txt| souboru.

Hlavní program je \verb|main.py| a pouští se dle zadání:
\begin{lstlisting}[language=bash]
python main.py -q <topics> -d <docs> -r <run> -o <out-file>
\end{lstlisting}

Pro mé vlastní pohodlí jsem si ještě napsal Makefile. Ta předpokládá, že:

\begin{itemize}
    \item Celá složka se zadáním a daty je v \verb|DATA_DIR|
    \item Program \verb|trec_eval| je k nalezení na cestě \verb|TREC_EVAL_BIN|
\end{itemize}

Obě proměnné jsou nastavitelné v prvních řádcích Makefile. Shrnutí co Makefile
umí:

\begin{lstlisting}[language=bash,texcl=true]
# Vygeneruje .res soubory pro defaultní run (run-0), oba jazyky a oba datové sety
make res

# Vygeneruje .res soubory pro run-1, oba jazyky a oba datové sety
make res run=run-1

# Vygeneruje .res soubory pro run-1, oba jazyky a train data
make res run=run-1 mode=train

# Vygeneruje .res soubor pro run-1, cs a train data
make res run=run-1 mode=train lan=cs

# Vygeneruje patřičné .res soubory (pokud je potřeba) a evaluuje je pomocí trec\_eval
make {eval} run=run-1 lan=en
\end{lstlisting}

\section{Experimenty}

\subsection{run-0}

Základní řešení se chová dle zadání:

\begin{enumerate}
    \item Z dokumentů vytáhne všechen text
    \item Rozdělí text všemi následujícími znaky: \verb|\t \n[](),.?!;:|
    \item Uloží do inverted indexu počet výskytů
    \item Z každé query si načte \verb|<title>| a stejně ho rozdělí na slova
    \item Vypočítá podobnost pomocí cosínu
\end{enumerate}

Zde jsou výsledky:

%TODO: gen tables script
%TODO: gen graphs script
\begin{center}
\begin{tabular}{c c c c c c} 
    run id & MAP & P@5 & P@30 & P@100 & P@500 \\ [0.5ex]
 \hline
    run-0\_cs & 0.1158 & 0.1920 & 0.0773 & 0.0324 & 0.0107\\
 \hline
\end{tabular}
\end{center}

\end{document}
